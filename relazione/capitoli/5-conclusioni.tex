\chapter{Conclusioni}

L’obiettivo di questo lavoro è stato lo studio e lo sviluppo di un sistema di controllo per una base robotica, che possa poi essere utilizzata per la ricerca in ambito di guida autonoma e percezione outdoor.

La base robotica era già presente in laboratorio IRA, tuttavia mancava dei componenti elettronici per poter essere utilizzabile.
Innanzitutto è stato quindi necessario fare un'analisi dei requisiti hardware, acquistare i componenti mancanti e procedere alla loro installazione fisica sulla base robotica.

È stato poi sviluppato il software di basso livello per interagire con i sensori e gli attuatori presenti sulla base robotica, in modo da poter controllare i motori e leggere le velocità delle ruote.

Ha avuto grande rilievo lo studio della comunicazione tra il microcontrollore e il computer di controllo, in modo da poter comandare il robot dal computer tramite il framework ROS. Sono stati sviluppati sia la parte di comunicazione a basso livello tra microcontrollore e computer, sia un nodo ROS in python in grado di ricavare l'odometria a partire dai dati degli encoder e di trasmettere comandi di velocità al robot.

Il tutto è stato poi testato usando un joypad collegato al computer per l'invio dei comandi al robot e visualizzando l'odometria ricavata tramite lo strumento RVIZ. 

Il sistema di controllo si è dimostrato funzionante, e Otto potrà quindi essere utilizzato in futuro, aggiungendo sensori eterocettivi, per esperimenti e ricerca all'interno del laboratorio.


