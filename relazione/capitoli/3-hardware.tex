\chapter{Base robotica Otto}

Otto è un robot a guida differenziale, progetto di ricerca del laboratorio di robotica IRALab, dell’Università degli Studi di Milano-Bicocca.

\section{Base robotica VolksBot RT 3}
VolksBot è un kit modulare per la costruzione di robot, progettato per il campo della ricerca e della  prototipazione rapida.
La base robotica è pensata per essere facilmente modificata ed adattata alle proprie esigenze in quanto composta da barre in alluminio combinabili fra di loro.

Nello specifico la base robotica utilizzata è composta da due ruote motrici frontali e una ruota basculante di supporto posteriore.
Ciascuna ruota motrice è collegata ad un motore a corrente continua combinato con una riduzione con un rapporto di trasmissione 1:74.
L'intero sistema robotico è alimentato tramite batterie a bordo del veicolo.
\begin{figure}[H]
\centering
\includegraphics[scale=0.45]{images/otto1.png}
\caption{Base robotica Otto}
\end{figure}

\section{Encoder}
Un encoder è dispositivo elettromeccanico in grado di convertire la posizione o il moto angolare in un codice digitale.

Nel nostro caso è stato montato un encoder in quadratura sull'asse del motore di ciascuna ruota motrice.
Questo tipo di sensore è formato da un LED, da una corona circolare con un pattern fisso che si ripete e da dei fotodiodi. 

\begin{figure}[H]
\centering
\includegraphics[scale=0.30]{images/corona.png}
\caption{Corona circolare dell'encoder}
\end{figure}

Il funzionamento si basa sulla capacità dei fotodiodi di percepire i cambiamenti di luce: il LED viene sempre alimentato ed emette quindi una luce costante, la corona dell'encoder gira insieme all'albero del motore e i fotodiodi rilevano i cambiamenti di luce dovuti al pattern sulla corona e tramite dei comparatori generano delle onde quadre.
Elaborando questi segnali è possibile misurare sia la distanza percorsa dalle ruote che la direzione del movimento.

\begin{figure}[H]
\hfill
\subfigure[Schema dell'encoder]{\includegraphics[scale=0.32]{images/encoder-1.png}}
\hfill
\subfigure[Segnale generato]{\includegraphics[scale=0.47]{images/quad-encoding-waveform-1.png}}
\hfill
\caption{Encoder in quadratura}
\end{figure}

\section{Motor driver}
Un motor driver è un dispositivo elettronico necessario per poter controllare i motori DC usando i segnali digitali generati da un microcontrollore.
Il circuito elettronico principale è chiamato ponte H e permette di controllare la polarità della tensione applicata a un carico. Tramite questo meccanismo siamo in grado sia di controllare la direzione del moto di un motore a corrente continua, stabilendo il verso nel quale fluisce la corrente, sia di regolare la velocità del motore modulando la tensione.

La modulazione della tensione avviene tramite un segnale PWM (pulse-width modulation). Questo segnale periodico ha due fasi: una in cui la tensione è alta (3.3V) e una fase in cui la tensione è bassa (0V). Il rapporto tra queste due fasi è detto duty-cycle e determina la tensione media in uscita.

\begin{figure}[H]
\centering
\includegraphics[width=\textwidth]{images/pwm.jpg}
\caption{Modulazione della tensione con segnale PWM.}
\end{figure}


Il motor driver scelto è Pololu Dual G2: questo dispositivo ha due circuiti H-bridge, così da poter controllare entrambi i motori in modo indipendente.
Per controllare ciascun motore sono presenti 3 segnali di input:
\begin{itemize}
    \item SLP: disabilita l'output se è uguale a 0
    \item PWM: input particolare rappresentabile come percentuale, modula la velocità
    \item DIR: imposta la direzione di azione del motore
\end{itemize}

Si hanno quindi le seguenti modalità di funzionamento:
\begin{table}[H]
    \centering
    \begin{tabular}{|l|l|l|l|}
    \hline
    SLP & DIR & PWM & Operazione                            \\ \hline
    1   & 0   & \%pwm & motore azionato in senso orario a velocità pwm \\
    1   & 1   & \%pwm & motore azionato in senso antiorario a velocità pwm         \\
    1   & x   & 0   & motore frenato                        \\
    0   & x   & x   & motore libero                     \\ \hline
    \end{tabular}
\end{table}

Sono inoltre presenti due segnali analogici di output per controllare il consumo dei motori (CS) e due segnali di output per monitorare lo stato di eventuali guasti (FLT).

\begin{figure}[H]
\centering
\includegraphics[scale=1.4]{images/pololu.png}
\caption{Schema delle connessioni del Pololu Dual G2.}
\end{figure}


\section{Microcontrollore}
La scheda di sviluppo scelta è una Nucleo STM32F767ZI.
Il processore è un Core Arm 32-bit Cortex-M7, ed è presente una Floating Point Unit per velocizzare le operazioni in virgola mobile.
La scheda ha 512kB di memoria RAM e 2MB di memoria FLASH.
È stata scelta per la grande disponibilità di periferiche integrate, in particolare per la realizzazione del sistema di controllo sono stati necessari: 
\begin{itemize}
    \item 2 timer a 32 bit per la gestione degli encoder.
    \item 1 timer a 16 bit per la generazione del segnale PWM.
    \item 1 periferica UART per la comunicazione con il computer.
\end{itemize}
Inoltre è presente un debugger integrato ST-LINK V2.1.

\begin{figure}[H]
\centering
\includegraphics[scale=0.65]{images/nucleo.png}
\caption{La scheda Nucleo. Evidenziato in rosso il debugger, in verde il processore ed in giallo i GPIO per interfacciarsi con le varie periferiche.}
\end{figure}

\section{Modulo FTDI}
Per quanto riguarda la comunicazione tra microcontrollore e computer si è scelto il protocollo UART. 
Per rendere il sistema utilizzabile collegando un qualsiasi computer è stato necessario un modulo apposito che convertisse il segnale UART in un segnale USB, così da non dover utilizzare una piattaforma apposita dotata di GPIO.

È stato utilizzato un modulo FTDI FT232RL in quanto ampiamente supportato a livello di driver Linux e perché, oltre alle linee di trasmissione e ricezione, presenta alcuni meccanismi di controllo del flusso hardware tramite appositi pin (CTS e RTS) e un pin per poter effettuare il reset del microcontrollore direttamente dal PC.


\begin{figure}[H]
\centering
\includegraphics[width=\textwidth]{images/infrastruttura.pdf}
\caption{Riepilogo generale dei componenti e delle loro connessioni.}
\end{figure}
